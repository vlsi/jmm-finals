% ---------------------------------------
% Memory chain. Пример возникновения.
%
\ifrender
\subsection{Memory chain}
\begin{frame}[fragile]{Memory chain: что это?}
\hbsprogresss{0}{0}{0}{0}{0}{1-}{0}{0}{0}%
\vskip0.25cm
\begin{minipage}[t]{.33\textwidth}
	\begin{lstlisting}[title=Thread 1]
T o = new T();
GL = o; ~\nd<1->[1,6]{w1}~
	\end{lstlisting}
\end{minipage}%
\begin{minipage}[t]{.33\textwidth}
	\begin{lstlisting}[title=Thread 2]
T o = GL; ~\nd<1->[1,2]{r1}~
GL2 = o; ~\nd<2->[2,3]{w2}~
\end{lstlisting}
\end{minipage}%
\begin{minipage}[t]{.33\textwidth}
	\begin{lstlisting}[title=Thread 3]
T o = GL2; ~\nd<3->[3-5]{r3}~
int r = o.x; ~\nd<4->[4-6]{r4}~
\end{lstlisting}
\end{minipage}

\vskipfill

% Стрелочки между tikz-якорями
\begin{tikzpicture}[overlay]
  \path[mc edge,->]<1-> (w1) edge [bend left=30] node {mc} (r1);
  \path[mc edge,->]<2-> (r1) edge [out=0, in=-45, looseness=3] node {mc} (w2);
  \path[mc edge,->]<3-> (w2) edge [bend left=30] node {mc} (r3);
  \path[dr edge,->]<4-> (r3) edge [out=45, in=45, looseness=2.5] node {dr} (r4);
  \path[mc edge,->]<5-> (r3) edge [out=235, in=235, looseness=2.5] node {mc} (r4);
  \path[mc edge,->]<6-> (w1) edge [out=-90, in=-90, looseness=0.4] node {mc} (r4);
\end{tikzpicture}

\begin{center}
\only<1>{Если чтение видит запись, то $w1 \mc r1$}%
\only<2>{Если пишем адрес созданного врагом объекта, то $r1 \mc w2$}%
\only<3>{\nd[3]{r3} видит \nd[3]{w2} $\Rightarrow w2 \mc r3$}%
\only<4,5>{$r3 \dr r4$ (читаем поле объекта) \uncover<5>{$\Rightarrow r3 \mc r4$}}%
\only<6>{{\mcbox} транзитивно (т.к. частичный порядок) $\Rightarrow w1 \mc r4$}%
\end{center}

\vskipfill

\end{frame}
\fi
