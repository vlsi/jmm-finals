% ---------------------------------------
% Пример 5 "reflection in action"
%
\ifrender
\subsection{Reflection in action (5)}
\begin{frame}[fragile]{Reflection in action (5)}
\hbsprogresss{0}{0}{0}{0}{0}{0}{0}{0}{0}%
% Сорцы и tikz-якоря
\begin{minipage}{0.5\textwidth}
	\begin{lstlisting}
T t = new T() {{
  fu = new U();
  fu.x = 1; ~\nd<1>[1]{w1}~
}}; ~\nd<4->[4]{f1}~
GLOBAL = t; ~\nd<6->[6]{a}~
U w = new U();
w.x = 42; ~\nd<1-6>[1,2,4]{w2}~
reflectSet(t.fu, w); ~\nd<3->[3]{f2}~
	\end{lstlisting}
\end{minipage}%
\begin{minipage}{0.5\textwidth}
	\begin{lstlisting}[title=Thread 2]
T t = GLOBAL;
if (t != null) {
  U u = t.fu;
  int result = u.x; ~\nd<1-6>[1,2,5]{r2}~
}
	\end{lstlisting}
\end{minipage}

\vskipfill

% Стрелочки между tikz-якорями
\begin{tikzpicture}[overlay]
  \path[hbs edge,maybe,->]<2-> (w2) edge [out = 45, in = 130, looseness = 1.4] node (hbs-w2-r2) {hb*?} (r2);
  \redcross<3->{f2}
  \redcross<4->{f1}
  \path[hb edge,maybe,->]<4-> (w2) edge [out=80, in=30, looseness=1.4] node (hb-w2-f1) {hb?} (f1);
  \redcross<4->{hb-w2-f1}
  \redcross<5->{hbs-w2-r2}
\end{tikzpicture}

\begin{center}
% Подписи, потихоньку появляющиеся под сорцами
\only<1>{Возможно ли в \result~получить $0$?}%
\only<2>{По-хорошему, нам нужно $w2 \hbs r2$ (иначе не запрещено видеть запись 0 в \lstinline!w.x!)}
\only<3>{\nd[3]{f2} не подходит для {\hbsbox}: нет подходящего $f2 \hb a$}%
\only<4>{\nd[4]{f1} тоже не подходит для {\hbsbox}: должно быть $w2 \hb f1$}%
\only<5>{Получается, что чтению \nd[5]{r2} не запрещено видеть 0}
\only<6>{Итого: после публикации менять final поля опасно\\\result~$\in \{0,1,42\}$}%
\end{center}

\vskipfill
\end{frame}
\fi
