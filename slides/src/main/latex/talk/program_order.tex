\ifrender
\subsection{Программный порядок}
\begin{frame}{Программный порядок (program order)}
\begin{itemize}[<+->]
\item Отражает то, в каком порядке написан исходный код
\item Компилятору запрещено переупорядочивать, игнорировать и менять операции, если это нарушит program order
\item Но это не значит, что всё выполняется именно так, как в исходном коде
\item Например: для операций над локальными переменными program order вообще не определён
\end{itemize}
\end{frame}
\fi
