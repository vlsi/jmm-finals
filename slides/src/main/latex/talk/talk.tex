\documentclass[russian,aspectratio=169,14pt]{beamer}
\usepackage[utf8]{inputenc}
\usepackage[T2A]{fontenc}
\usepackage{listings}
\usepackage{tikz}
\usepackage{amsmath}
\usetikzlibrary{arrows,shapes}

\definecolor{mygreen}{rgb}{0,0.4,0}
\definecolor{myid}{rgb}{0.1,0.1,0.1}
\lstdefinestyle{Java}{language=java,
        numbers=none,stepnumber=1,numberstyle=\small\ttfamily,
        numbersep=5pt,extendedchars=true,
        commentstyle=\color{mygreen}\ttfamily,
        stringstyle=\color{magenta},
        keywordstyle=\color{violet}\bfseries,
        ndkeywordstyle=\color{yellow}\bfseries,
        identifierstyle=\color{myid},
        basicstyle=\small\ttfamily,
        escapechar=~,
        frame=none,
        lineskip=5pt,
        tabsize=2
}

\lstset{breakatwhitespace=false,
        language=Java,
        style=Java,
}

\definecolor{gold}{rgb}{0.85,.66,0}
\selectcolormodel{cmyk}
\definecolor{c1}{tHsb}{0,0.5,1}
\definecolor{c2}{tHsb}{90,0.5,1}
\definecolor{c3}{tHsb}{180,0.5,1}
\definecolor{c4}{tHsb}{270,0.5,1}
%\convertcolorspec{tHsb}{210,1,1}{rgb}{\c1spec}
%\definecolor{c1}{rgb}{\c1spec}	

\tikzstyle{labelfont} = [font=\ttfamily\small\bfseries]
\tikzstyle{na} = [anchor=base,baseline,rectangle,draw,labelfont]
\tikzstyle{main node} = [na,fill=black!20]
\tikzstyle{target node} = [na,fill=gold!20]
%\tikzstyle{hb label} = [fill=green!30,labelfont]
\tikzstyle{hb label} = [fill=c1,labelfont]
\tikzstyle{mc label} = [fill=c2,labelfont]
\tikzstyle{dr label} = [fill=c3,labelfont]
\tikzstyle{hbs label} = [fill=c4,labelfont]
\tikzstyle{edge style} = [thick,line width=2,labelfont]
\tikzstyle{hb edge} = [draw=c1,edge style]
\tikzstyle{mc edge} = [draw=c2,edge style]
\tikzstyle{dr edge} = [draw=c3,edge style]
\tikzstyle{hbs edge} = [draw=c4,edge style]
\tikzstyle{maybe} = [dashed]

\tikzset{alt/.code args={#1#2#3}{%
\if\relax#1\relax\pgfkeysalso{#3}\else\alt<#1>{\pgfkeysalso{#2}}{\pgfkeysalso{#3}}\fi%
}}	

\newcommand{\hb}{\xrightarrow{\fcolorbox{green!30}{green!30}{\scriptsize hb}}}
\newcommand{\mc}{\xrightarrow{\fcolorbox{red!30}{red!30}{\scriptsize mc}}}
\newcommand{\dr}{\xrightarrow{\fcolorbox{blue!30}{blue!30}{\scriptsize dr}}}
\newcommand{\hbs}{\xrightarrow{\fcolorbox{gold!30}{gold!30}{\scriptsize $hb^*$}}}
\newcommand{\result}{\lstinline!result!}

\newcommand<>\nd[2][]{%
\ifx\relax#3\relax\tikz[na]\node[alt={#1}{target node}{main node}] {#2};%
\else\tikz[na]\node#3[alt={#1}{target node}{main node}] (#2) {#2};\fi%
}

\newcommand<>{\redcross}[1]{
\ifx\relax#2\relax%
\draw[red, line width=0.3mm]
    (#1.south west) -- (#1.north east)
    (#1.south east) -- (#1.north west);%
\else%
\draw#2[red, line width=0.3mm]
    (#1.south west) -- (#1.north east)
    (#1.south east) -- (#1.north west);%
\fi%
}

\newcommand<>\dimunless[1]{\alt#2{\textcolor{black}{#1}}{\textcolor{black!20}{#1}}}

\newcommand{\hbsprogresss}[9]{%
\begin{tikzpicture}[overlay]%
\node [anchor=north east, inner sep=10pt] at (current page.north east){%
$\dimunless<#1>{w}~\dimunless<#2>{\xrightarrow{hb}}~\dimunless<#3>{f}~\dimunless<#4>{\xrightarrow{hb}}~\dimunless<#5>{a}~\dimunless<#6>{\xrightarrow{mc}}~\dimunless<#7>{r1}~\dimunless<#8>{\xrightarrow{dr}}~\dimunless<#9>{r2}$%
};%
\end{tikzpicture}%
}

\newcommand{\hbsprogress}[4]{\hbsprogresss{#1}{#1}{#1}{#2}{#2}{#3}{#3}{#4}{#4}}

\newcommand\vskiptitle{\vskip0.25cm}
\newcommand\vskipfill{\vskip0pt plus 1filll}

\tikzstyle{every picture}+=[remember picture]

% Позволяет отключить 
%\let\ifrender\iftrue
\let\ifrender\iffalse

\begin{document}

% ---------------------------------------
% Dereference chain. Пример возникновения
%
\ifrender
\begin{frame}[fragile]
\frametitle{Dereference chain: что это?}
\vskiptitle
\begin{center}
\begin{minipage}[3]{.5\linewidth}
	\begin{lstlisting}[escapechar=~]
T local = GLOBAL; ~\nd<1->{r1}~

int localX = local.x; ~\nd<1->[1-]{r2}~
	\end{lstlisting}
\end{minipage}
\end{center}
\vskipfill

\begin{itemize}[<+->]
\item \nd[1-]{r2} читает поле объекта
\item Поток не создавал объект
\item Значит, где-то мы должны были читать адрес этого объекта
\item Это называется $r1 \dr r2$
\end{itemize}
\vskipfill

% Стрелочки между tikz-якорями
\begin{tikzpicture}[overlay]
  \path[dr edge,->]<4-> (r1) edge [bend right, out = 90, in = 90, looseness = 2] node[dr label] {dr} (r2);
\end{tikzpicture}%
\end{frame}
\fi

% ---------------------------------------
% Dereference chain. Между потоками не возникает
%
\ifrender
\begin{frame}[fragile]
\frametitle{Dereference chain между потоками}
\vskiptitle
\begin{minipage}[t]{.49\textwidth}
	\begin{lstlisting}[escapechar=~,title=Thread 1]
T localA = new T(); ~\nd<2->[2,3]{a}~
GLOBAL = localA;
	\end{lstlisting}
\end{minipage}
\begin{minipage}[t]{.49\textwidth}
	\begin{lstlisting}[escapechar=~,title=Thread 2]
T localB = GLOBAL; ~\nd<1->[1]{r1}~
if (localB != null) {
  int localX = localB.x; ~\nd<1->[1-]{r2}~
}
\end{lstlisting}
\end{minipage}
\vskipfill

\begin{center}
\only<1>{$r1 \dr r2$ (читаем поле несозданного нами объекта)}%
\only<2>{Есть ли $a \dr r2$?}%
\only<3>{Между потоками $\dr$ не возникает!}%
\end{center}
\vskipfill

% Стрелочки между tikz-якорями
\begin{tikzpicture}[overlay]
  \path[dr edge,->]<1-> (r1) edge [bend right, out = 90, in = 90, looseness = 2] node[dr label] {dr} (r2);
  \path[dr edge,maybe,->]<2-> (a) edge [bend right, out = 270, in = 270, looseness = 0.8] node [dr label] (drq) {dr?} (r2);
  \redcross<3->{drq}
\end{tikzpicture}
\end{frame}
\fi

% ---------------------------------------
% Dereference chain. Контрольный выстрел
%
\ifrender
\begin{frame}[fragile]
\frametitle{Dereference chain: контрольный выстрел}
\vskiptitle
\begin{center}
\begin{minipage}{.5\linewidth}
	\begin{lstlisting}[escapechar=~]
T local = GLOBAL; ~\nd<1->[2]{ra}~

local = GLOBAL; ~\nd<1->[2]{rb}~

int localX = local.x; ~\nd<1->{r2}~
	\end{lstlisting}
\end{minipage}
\end{center}
\vskipfill

\begin{center}
\only<1>{Есть ли здесь $\dr$?}%
\only<2>{Один из $ra \dr r2$ или $rb \dr r2$ точно должен быть, но точнее сказать невозможно}%
\end{center}
\vskipfill

% Стрелочки между tikz-якорями
\begin{tikzpicture}[overlay]
  \path[dr edge,maybe,->]<2-> (ra) edge [bend right, out = 90, in = 90, looseness = 2] node[dr label] {dr?} (r2);
  \path[dr edge,maybe,->]<2-> (rb) edge [bend right, out = 90, in = 90, looseness = 1] node[dr label] {dr?} (r2);
\end{tikzpicture}

\end{frame}
\fi
% ---------------------------------------
% TODO TODO: нужен какой-то простой пример
%

% ---------------------------------------
% Memory chain. Пример возникновения.
%
\ifrender
\begin{frame}[fragile]
\frametitle{Memory chain: что это?}
\vskip0.25cm
\begin{minipage}[t]{.33\textwidth}
	\begin{lstlisting}[escapechar=~,title=Thread 1]
T o = new T();
GL = o; ~\nd<1->[1,6]{w1}~
	\end{lstlisting}
\end{minipage}%
\begin{minipage}[t]{.33\textwidth}
	\begin{lstlisting}[escapechar=~,title=Thread 2]
T o = GL; ~\nd<1->[1,2]{r1}~
GL2 = o; ~\nd<2->[2,3]{w2}~
\end{lstlisting}
\end{minipage}%
\begin{minipage}[t]{.33\textwidth}
	\begin{lstlisting}[escapechar=~,title=Thread 3]
T o = GL2; ~\nd<3->[3,5]{r3}~
int r = o.x; ~\nd<4->[4-6]{r4}~
\end{lstlisting}
\end{minipage}
\vskipfill
\begin{center}
\only<1>{Если чтение видит запись, то $w1 \mc r1$}%
\only<2>{Если пишем адрес созданного ненами объекта, то $r1 \mc w2$}%
\only<3>{\nd[3]{r3} видит \nd[3]{w2} $\Rightarrow w2 \mc r3$}%
\only<4,5>{$r3 \dr r4$ (читаем поле объекта) \uncover<5>{$\Rightarrow r3 \mc r4$}}%
\only<6>{$\mc$ транзитивно (т.к. частичный порядок) $\Rightarrow w1 \mc r4$}%
\end{center}

\vskipfill

% Стрелочки между tikz-якорями
\begin{tikzpicture}[overlay]
  \path[mc edge,->]<1-> (w1) edge [bend left=30] node[mc label] {mc} (r1);
  \path[mc edge,->]<2-> (r1) edge [out=0, in=-45, looseness=3] node[mc label] {mc} (w2);
  \path[mc edge,->]<3-> (w2) edge [bend left=30] node[mc label] {mc} (r3);
  \path[dr edge,->]<4-> (r3) edge [out=45, in=45, looseness=2.5] node[dr label] {dr} (r4);
  \path[mc edge,->]<5-> (r3) edge [out=235, in=235, looseness=2.5] node[mc label] {mc} (r4);
  \path[mc edge,->]<6-> (w1) edge [out=-90, in=-90, looseness=0.4] node[mc label] {mc} (r4);
\end{tikzpicture}
\end{frame}
\fi

% ---------------------------------------
% HB*
%
\ifrender
\begin{frame}[fragile]
\frametitle{Семантика final полей одним слайдом}
\vskipfill
\scalebox{1.5}{$w \hb f \hb a \mc r1 \dr r2 \Rightarrow w \hbs r2$}
\vskipfill
\begin{itemize}[<+->]
\item Если единственный путь от записи к чтению идёт через все эти стрелочки, то мы не можем увидеть более ранние записи
\item Если путей больше одного -- мы попали
\end{itemize}
\vskipfill
\end{frame}
\fi

% ---------------------------------------
% Пример 1 "простое final поле"
%
\ifrender
\begin{frame}[fragile]
\frametitle{Эталонный final (1)}
\hbsprogress{2-}{2-}{6-}{7-}%
\vskiptitle
% Сорцы и tikz-якоря
\begin{minipage}{0.5\textwidth}
	\begin{lstlisting}[escapechar=~,title=Thread 1]
T l = new T() {{
  fx = 42; ~\nd<1->[1,2,4,8,9]{w}~
}}; ~\nd<2->[2]{f}~
GLOBAL = l; ~\nd<2->[2,3,6]{a}~
	\end{lstlisting}
\end{minipage}%
\begin{minipage}{0.5\textwidth}
	\begin{lstlisting}[escapechar=~,title=Thread 2]
T o = GLOBAL; ~\nd<3-6>[3,4,5]{r0}~
if (o != null) {
  int result = o.fx; ~\nd<4->[4,5,6,7]{r1}~ ~\nd<1,7->[1,7,8,9]{r2}~
}
	\end{lstlisting}
\end{minipage}

% Подписи, потихоньку появляющиеся под сорцами
\begin{center}
\only<1>{Какое значение \result~допустимо? $\{0, 42, NPE\}$?}%
\only<2>{Действия в одном потоке образуют happens-before:\\ $w \hb f$, $f \hb a$}%
\only<3>{\nd[3]{rg} видит запись \nd[3]{a}:\\ $a \mc r0$}%
\only<4>{Поток 2 не создавал объект, \nd[4]{r1} читает его поле, а \nd[4]{r} это единственное чтение адреса объекта, поэтому dereference chain:\\ $rg \dr r1$}%
\only<5>{$r0 \dr r1 \Rightarrow r0 \mc r1$}%
\only<6>{$a \mc r1$ ($\mc$ транзитивно)}%
\only<7>{Возьмём $r2 \equiv r1$, тогда $r1 \dr r2$ ($\dr$ рефлексивно)}%
\only<8>{Нашли всё необходимое для $HB^*$: \\$w \hb f \hb a \mc r1 \dr r2 \Rightarrow w \hbs r2$}%
\only<9>{$(w \hbs r2) \Rightarrow$ \result~$\in \{42\}$}%
\end{center}

\vskipfill

% Стрелочки между tikz-якорями
\begin{tikzpicture}[overlay]
  \path[hb edge,->]<2-> (w) edge [out=120, in=110, looseness=2.5] node[hb label] {hb} (f);
  \path[hb edge,->]<2-> (f) edge [bend left=15] node[hb label] {hb} (a);
  \path[mc edge,->]<3-6> (a) edge [bend left=40, in=140] node[mc label] {mc} (r0);
  \path[dr edge,->]<4-5> (r0) edge [bend right, out = 270, in = 270, looseness = 2] node[dr label] {dr} (r1);
  \path[mc edge,->]<5,6> (r0) edge [bend left=20] node[mc label] {mc} (r1);
  \path[mc edge,->]<6> (a) edge [bend left=70, in=90, looseness=1.2] node[mc label] {mc} (r1);
  \path[mc edge,->]<7-> (a) edge [bend left=45, in=135] node[mc label] {mc} (r1);
  \path[dr edge,->]<7-> (r1) edge [out=-90, in=-90, looseness = 1.5] node[dr label] {dr} (r2);
  \path[hbs edge,->]<8-> (w) edge [bend right, out = 45, in = 110, looseness = 1.0] node[hbs label] {hb*} (r2);
\end{tikzpicture}
\end{frame}
\fi

% ---------------------------------------
% Пример 2 "массив внутри final поля"
%
\ifrender
\begin{frame}[fragile]
\frametitle{Массив внутри final (2)}
\vskiptitle
\hbsprogress{2-}{2-}{3-}{4-}%
% Сорцы и tikz-якоря
\begin{minipage}{0.5\textwidth}
	\begin{lstlisting}[escapechar=~]
T l = new T() {{
  int[] u = new int[1];
  u[0] = 42; ~\nd<1->[1,2,5,6]{w}~
  fx = u;
}}; ~\nd<2->[2]{f}~
GLOBAL = l; ~\nd<2->[2,3]{a}~
	\end{lstlisting}
\end{minipage}%
\begin{minipage}{0.5\textwidth}
	\begin{lstlisting}[escapechar=~]
T o = GLOBAL;
if (o != null) {
  int[] lfx = o.fx; ~\nd<3->[3,4]{r1}~
  int result = lfx[0]; ~\nd<1->[1,4,5,6]{r2}~
}
	\end{lstlisting}
\end{minipage}

% Подписи, потихоньку появляющиеся под сорцами
\begin{center}
\only<1>{Какое значение \result~допустимо? $\{0, 42\}$?}%
\only<2>{Действия в одном потоке образуют happens-before:\\ $w \hb f$, $f \hb a$}%
\only<3>{\nd[3]{r1} видит запись \nd[3]{a} (т.к. простое final поле):\\ $a \mc r1$}%
\only<4>{Поток 2 не создавал массив, $r2$ читает его элемент, $r1$ это единственное чтение адреса массива, поэтому dereference chain:\\ $r1 \dr r2$}%
\only<5>{Нашли всё необходимое для $HB^*$: \\$w \hb f \hb a \mc r1 \dr r2 \Rightarrow w \hbs r2$}%
\only<6>{($w \hbs r2) \Rightarrow$ \result~$\in \{42\}$}%
\end{center}

\vskipfill

% Стрелочки между tikz-якорями
\begin{tikzpicture}[overlay]
  \path[hb edge,->]<2-> (w) edge [] node[hb label] {hb} (f);
  \path[hb edge,->]<2-> (f) edge [bend left=15] node[hb label] {hb} (a);
  \path[mc edge,->]<3-> (a) edge [bend left, looseness=0.7] node[mc label] {mc} (r1);
  \path[dr edge,->]<4-> (r1) edge [bend right, out = 270, in = 270, looseness = 2.2] node[dr label] {dr} (r2);
  \path[hbs edge,->]<5-> (w) edge [bend right, out = 90, in = 90, looseness = 0.7] node[hbs label] {hb*} (r2);
\end{tikzpicture}
\end{frame}
\fi

% ---------------------------------------
% Пример 2.1 "порядок присваивания final"
%
\ifrender
\begin{frame}[fragile]
\frametitle{Массив наоборот (2.1)}
\vskiptitle
\hbsprogress{2-}{2-}{2-}{2-}%
\begin{minipage}{0.5\textwidth}
	\begin{lstlisting}[escapechar=~]
T l = new T() {{
  int[] u = new int[1];
  fx = u; // (!)
  u[0] = 42; ~\nd<1->[1,2,3]{w}~
}}; ~\nd<2->{f}~
GLOBAL = l; ~\nd<2->{a}~
	\end{lstlisting}
\end{minipage}%
\begin{minipage}{0.5\textwidth}
	\begin{lstlisting}[escapechar=~]
T o = GLOBAL;
if (o != null) {
  int[] lfx = o.fx; ~\nd<2->{r1}~
  int result = lfx[0]; ~\nd<2->[2,3]{r2}~
}
	\end{lstlisting}
\end{minipage}

% Подписи, потихоньку появляющиеся под сорцами
\begin{center}
\only<1>{Какое значение \result~допустимо? $\{0, 42\}$?}%
\only<2>{Строим $\hbs$ так же как и в предыдущем случае}%
\only<3>{($w \hbs r2) \Rightarrow$ \result~$\in \{42\}$\\Результат не зависит от порядка записи final полей!}%
\end{center}

\vskipfill

% Стрелочки между tikz-якорями
\begin{tikzpicture}[overlay]
  \path[hb edge,->]<2-> (w) edge [] node[hb label] {hb} (f);
  \path[hb edge,->]<2-> (f) edge [bend left=15] node[hb label] {hb} (a);
  \path[mc edge,->]<2-> (a) edge [bend left, looseness=0.7] node[mc label] {mc} (r1);
  \path[dr edge,->]<2-> (r1) edge [bend right, out = 270, in = 270, looseness = 2.2] node[dr label] {dr} (r2);
  \path[hbs edge,->]<2-> (w) edge [bend right, out = 90, in = 90, looseness = 0.7] node[hbs label] {hb*} (r2);
\end{tikzpicture}
\end{frame}
\fi

% ---------------------------------------
% Пример 3 "утекание this из конструктора"
%
\ifrender
\begin{frame}[fragile]
\frametitle{Утекание this (3)}
\vskiptitle
\hbsprogress{2-}{0}{0}{0}%
% Сорцы и tikz-якоря
\begin{minipage}{0.5\textwidth}
	\begin{lstlisting}[escapechar=~,title=Thread 1]
T l = new T() {{
  fx = 42; ~\nd<1->[1,2]{w}~
  GLOBAL = this; ~\nd<2->[2,3,4,5]{a}~
}}; ~\nd<2->[2,3,4]{f}~
	\end{lstlisting}
\end{minipage}%
\begin{minipage}{0.5\textwidth}
	\begin{lstlisting}[escapechar=~,title=Thread 2]
T o = GLOBAL;
if (o != null) {
  int result = o.fx; ~\nd<1->[1,5]{r2}~
}
	\end{lstlisting}
\end{minipage}

% Подписи, потихоньку появляющиеся под сорцами
\begin{center}
\only<1>{Какое значение \result~допустимо? $\{0, 42\}$?}%
\only<2>{Действия в одном потоке образуют happens-before:\\ $w \hb f$, $a \hb f$}%
\only<3>{Но нам-то нужно $f \hb a$!}%
\only<4>{Если $a \hb f$ и $f \hb a$, то $a \equiv f$ (антисимметричность $\hb$)\\Но публикация ссылки это никак не freeze action!}%
\only<5>{Нет $\hbs$, поэтому возможны все варианты: \result~$\in \{0, 42\}$}%
\end{center}

\vskipfill

% Стрелочки между tikz-якорями
\begin{tikzpicture}[overlay]
  \path[hb edge,->]<2-> (w) edge [out=135, in=135, looseness=1.5] node[hb label] {hb} (f);
  \path[hb edge,->]<2-> (a) edge [bend left=60] node[hb label] {hb} (f);
  \path[hb edge,maybe,->]<3-> (f) edge [bend right=5] node[hb label] (hbq) {hb?} (a);
  \redcross<4->{hbq}
\end{tikzpicture}
\end{frame}
\fi

% ---------------------------------------
% Пример 4 "утекание this из конструктора, чтение из другой ссылки"
%
\iftrue
\begin{frame}[fragile]
\frametitle{Двойное утекание this (4)}
\vskiptitle
\hbsprogress{2-}{2-}{0}{5-}%
% Сорцы и tikz-якоря
\begin{minipage}{0.5\textwidth}
	\begin{lstlisting}[escapechar=~,title=Thread 1]
T l = new T() {{
  fx = 42; ~\nd<1->[1,2]{w}~
  GLb = this; ~\nd<2->[3,6]{b}~
}}; ~\nd<2->[2]{f}~
GLa = l; ~\nd<2->[2,3]{a}~
	\end{lstlisting}
\end{minipage}%
\begin{minipage}{0.5\textwidth}
	\begin{lstlisting}[escapechar=~,title=Thread 2]
T u = GLb; ~\nd<3->[3,4,5,6]{rb}~
T o = GLa; ~\nd<3->[3,4]{ra}~
if (o != null) {
  int result = o.fx; ~\nd<4->[4,5,6]{r1}~ ~\nd<1>[1]{r2}~
}
	\end{lstlisting}
\end{minipage}

% Подписи, потихоньку появляющиеся под сорцами
\begin{center}
\only<1>{Какое значение \result~допустимо? $\{0, 42\}$?}%
\only<2>{Действия в одном потоке образуют happens-before:\\ $w \hb f$, $f \hb a$}%
\only<3>{NPE нам неинтересны, поэтому $b \mc rb$ и $a \mc ra$!}%
\only<4>{Где-то должно быть $\dr$: $rb \dr r1$ или $ra \dr r1$}%
\only<5>{Если $rb \dr r1$, то мы попали, т.к. $\hbs$ не строится}%
\only<6>{Вывод: если наш поток уже читал объект с незамороженными полями, то гараний final semantics нет!}%
\end{center}

\vskipfill

% Стрелочки между tikz-якорями
\begin{tikzpicture}[overlay]
  \path[hb edge,->]<2-> (w) edge [out=120, in=110, looseness=2.5] node[hb label] {hb} (f);
  \path[hb edge,->]<2-> (f) edge [bend right=80, looseness=1.6] node[hb label] {hb} (a);
  \path[hb edge,maybe,->]<2-> (f) edge [bend right=5] node[hb label] (hbb) {hb?} (b);
  \redcross<2->{hbb}
  \path[mc edge,->]<3-> (b) edge [bend left, looseness=1.2] node[mc label] {mc} (rb);
  \path[mc edge,->]<3-> (a) edge [bend left, looseness=0.7] node[mc label] {mc} (ra);
  \path[dr edge,maybe,->]<4-> (ra) edge [out = -90, in = 235, looseness = 1.5] node[dr label] {dr?} (r1);
  \path[dr edge,maybe,->]<4> (rb) edge [out = 0, in = 90, looseness = 1.5] node[dr label] {dr?} (r1);
  \path[dr edge,->]<5-> (rb) edge [out = 0, in = 90, looseness = 1.5] node[dr label] {dr} (r1);
\end{tikzpicture}
\end{frame}
\fi

% ---------------------------------------
% Пример 5 "reflection in action"
%
\ifrender
\begin{frame}[fragile]
\frametitle{Reflection in action (5)}
\hbsprogresss{0}{0}{4-}{4-}{2-}{2-}{2-}{0}{0}%
% Сорцы и tikz-якоря
\begin{minipage}{0.5\textwidth}
	\begin{lstlisting}[escapechar=~]
T t = new T() {{
  fu = new U();
  fu.x = 1; ~\nd<1>[1]{w1}~
}}; ~\nd<4->[4,6,7]{f1}~
GLOBAL = t; ~\nd<2,8>[2,7]{a}~
U w = new U();
w.x = 42; ~\nd<1,3->[1,3,6,7]{w2}~
reflectSet(t.fu, w); ~\nd<2->[2,4,5,8]{b}~ ~\nd<4->[4,5]{f2}~
	\end{lstlisting}
\end{minipage}%
\begin{minipage}{0.5\textwidth}
	\begin{lstlisting}[escapechar=~,title=Thread 2]
T t = GLOBAL;
if (t != null) {
  U u = t.fu; ~\nd<2->[2]{r1}~
  int result = u.x; ~\nd<1,3->[1,3]{r2}~
}
	\end{lstlisting}
\end{minipage}

% Подписи, потихоньку появляющиеся под сорцами
\begin{center}
\only<1>{Какое значение \result~допустимо? $\{0, 1, 42\}$?}%
\only<2>{Предположим \nd[2]{r1} видит \nd[2]{b}: $b \mc r1$}%
\only<3>{По-хорошему, нам нужно $w2 \hbs r2$}
\only<4,5>{Попробуем найти f: $f \hb b$. На эту роль подходит только \nd[4,5]{f1}}%
\only<5>{\\Ведь freeze происходит после измерения final полей}%
\only<6>{Если f это f1, то должно быть $w2 \hb f1$}%
\only<7>{Program order наносит ответный удар: $f1 \hb w2$}%
\only<8>{Итого: после публикации менять final поля опасно. \result~$\in \{0,1,42\}$}%
\end{center}

\vskipfill

% Стрелочки между tikz-якорями
\begin{tikzpicture}[overlay]
  \path[mc edge,->]<2-> (b) edge [out=-45, in=-135, looseness=1.2] node[mc label] {mc} (r1);
  \path[hbs edge,maybe,->]<3-> (w2) edge [out = 45, in = 130, looseness = 1.4] node[hbs label] {hb*?} (r2);
  \path[hb edge,->]<4-> (f1) edge [out=-100, in=-145, looseness=1.45] node[hb label] {hb} (b);
  \path[hb edge,maybe,->]<4-> (f2) edge [out=90, in=90, looseness=3] node[hb label] (hb-f2-b) {hb?} (b);
  \redcross<5->{hb-f2-b}
  \path[hb edge,maybe,->]<6-> (w2) edge [out=80, in=20, looseness=1.4] node[hb label] (hb-w2-f1) {hb?} (f1);
  \redcross<7->{hb-w2-f1}
\end{tikzpicture}
\end{frame}
\fi

\end{document}
