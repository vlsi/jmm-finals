\documentclass[russian,aspectratio=169,14pt]{beamer}

\usepackage[utf8]{inputenc}
\usepackage[T2A]{fontenc}
\usepackage{listings}
\usepackage{tikz}
\usepackage{amsmath}
\usetikzlibrary{arrows,shapes}

\definecolor{mygreen}{rgb}{0,0.4,0}
\definecolor{myid}{rgb}{0.1,0.1,0.1}
\lstdefinestyle{Java}{language=java,
        numbers=none,stepnumber=1,numberstyle=\small\ttfamily,
        numbersep=5pt,extendedchars=true,
        commentstyle=\color{mygreen}\ttfamily,
        stringstyle=\color{magenta},
        keywordstyle=\color{violet}\bfseries,
        ndkeywordstyle=\color{yellow}\bfseries,
        identifierstyle=\color{myid},
        basicstyle=\small\ttfamily,
        escapechar=~,
        frame=none,
        lineskip=5pt,
        tabsize=4
}

\lstset{breakatwhitespace=false,
        language=Java,
        style=Java,
}

\definecolor{gold}{rgb}{0.85,.66,0}

\tikzstyle{na} = [baseline=-0.5ex]
\tikzstyle{main node} = [rectangle,fill=black!20,draw,font=\ttfamily\small\bfseries]
\tikzstyle{target node} = [rectangle,fill=gold!20,draw,font=\ttfamily\small\bfseries]
\tikzstyle{hb label} = [fill=green!30,font=\ttfamily\small\bfseries]
\tikzstyle{mc label} = [fill=red!30,font=\ttfamily\small\bfseries]
\tikzstyle{dr label} = [fill=blue!30,font=\ttfamily\small\bfseries]
\tikzstyle{dr? label} = [fill=blue!30,font=\ttfamily\small\bfseries]
\tikzstyle{hbs label} = [fill=gold!30,font=\ttfamily\small\bfseries]
\tikzstyle{hb edge} = [draw=green!30,thick,line width=2,font=\ttfamily\small\bfseries]
\tikzstyle{mc edge} = [draw=red!30,thick,line width=2,font=\ttfamily\small\bfseries]
\tikzstyle{dr edge} = [draw=blue!30,thick,line width=2,font=\ttfamily\small\bfseries]
\tikzstyle{dr? edge} = [draw=blue!30,thick,dashed,line width=2,font=\ttfamily\small\bfseries]
\tikzstyle{hbs edge} = [draw=gold!30,thick,line width=2,font=\ttfamily\small\bfseries]

\newcommand{\hb}{\xrightarrow{\fcolorbox{green!30}{green!30}{\scriptsize hb}}}
\newcommand{\mc}{\xrightarrow{\fcolorbox{red!30}{red!30}{\scriptsize mc}}}
\newcommand{\dr}{\xrightarrow{\fcolorbox{blue!30}{blue!30}{\scriptsize dr}}}
\newcommand{\hbs}{\xrightarrow{\fcolorbox{gold!30}{gold!30}{\scriptsize $hb^*$}}}

\tikzstyle{every picture}+=[remember picture]

\begin{document}

% ---------------------------------------
% Dereference chain. Пример возникновения
%
\begin{frame}[fragile]
\frametitle{Dereference chain: что это?}
\vskip0.5cm
\begin{center}
\begin{minipage}[3]{.5\linewidth}
	\begin{lstlisting}[escapechar=~]
T local = GLOBAL; ~\tikz[na]\node[main node] (r1) {r1};~

int localX = local.x; ~\tikz[na]\node[target node] (r2) {r2};~
	\end{lstlisting}
\end{minipage}
\end{center}
\vskip0pt plus 1filll
\begin{center}
\only<1>{$r2$ читает поле объекта $GLOBAL$}%
\only<2>{Поток не создавал объект}%
\only<3>{Значит, где-то мы должны были читать адрес объекта $GLOBAL$}%
\only<4>{Это называется $r1 \dr r2$}%
\end{center}

% Have no idea what it is, but it avoids repositioning of code blocks across slides
\vskip0pt plus 1filll

% Стрелочки между tikz-якорями
\begin{tikzpicture}[overlay]
  \path[dr edge,->]<4-> (r1) edge [bend right, out = 90, in = 90, looseness = 2] node[dr label] {dr} (r2);
\end{tikzpicture}

\end{frame}

% ---------------------------------------
% Dereference chain. Между потоками не возникает
%
\begin{frame}[fragile]
\frametitle{Dereference chain между потоками}
\vskip0.25cm
\begin{minipage}[t]{.49\textwidth}
	\begin{lstlisting}[escapechar=~,title=Thread 1]
T localA = new T(); ~\tikz[na]\node<2->[main node] (a) {a};~
GLOBAL = localA;
	\end{lstlisting}
\end{minipage}
\begin{minipage}[t]{.49\textwidth}
	\begin{lstlisting}[escapechar=~,title=Thread 2]
T localB = GLOBAL; ~\tikz[na]\node[main node] (r1) {r1};~
if (localB != null) {
  int localX = localB.x; ~\tikz[na]\node[target node] (r2) {r2};~
}
\end{lstlisting}
\end{minipage}
\vskip0pt plus 1filll
\begin{center}
\only<1>{$r1 \dr r2$ (читаем поле несозданного нами объекта)}%
\only<2>{Есть ли $a \dr r2$?}%
\only<3>{Между потоками $\dr$ не возникает!}%
\end{center}

% Have no idea what it is, but it avoids repositioning of code blocks across slides
\vskip0pt plus 1filll

% Стрелочки между tikz-якорями
\begin{tikzpicture}[overlay]
  \path[dr edge,->]<1-> (r1) edge [bend right, out = 90, in = 90, looseness = 2] node[dr label] {dr} (r2);
  \path[dr? edge,->]<2-> (a) edge [bend right, out = 270, in = 270, looseness = 0.8] node [dr? label] (drq) {dr?} (r2);
  \draw<3->[red, line width=1mm]
    (drq.south west) -- (drq.north east)
    (drq.south east) -- (drq.north west);
\end{tikzpicture}

\end{frame}

% ---------------------------------------
% Dereference chain. Контрольный выстрел
%
\begin{frame}[fragile]
\frametitle{Dereference chain: контрольный выстрел}

\begin{center}
\begin{minipage}{.5\linewidth}
	\begin{lstlisting}[escapechar=~]
T local = GLOBAL; ~\tikz[na]\node[main node] (ra) {ra};~

local = GLOBAL; ~\tikz[na]\node[main node] (rb) {rb};~

int localX = local.x; ~\tikz[na]\node[target node] (r2) {r2};~
	\end{lstlisting}
\end{minipage}
\end{center}
\vskip0pt plus 1filll
\begin{center}
\only<1>{Есть ли здесь $\dr$?}%
\only<2>{Один из $ra \dr r2$ или $rb \dr r2$ точно должен быть, но точнее сказать невозможно}%
\end{center}

% Have no idea what it is, but it avoids repositioning of code blocks across slides
\vskip0pt plus 1filll

% Стрелочки между tikz-якорями
\begin{tikzpicture}[overlay]
  \path[dr? edge,->]<2-> (ra) edge [bend right, out = 90, in = 90, looseness = 2] node[dr? label] {dr?} (r2);
  \path[dr? edge,->]<2-> (rb) edge [bend right, out = 90, in = 90, looseness = 1] node[dr? label] {dr?} (r2);
\end{tikzpicture}

\end{frame}

% ---------------------------------------
% TODO TODO: нужен какой-то простой пример
%


% ---------------------------------------
% Пример 3 "массив внутри final поля"
%
\begin{frame}[fragile]
\frametitle{Массив в final поле (пример 3)}

\begin{tikzpicture}[overlay]
\node [anchor=north east, inner sep=10pt] at (current page.north east){%
\only<1>{$\textcolor{black!20}{w \xrightarrow{hb} f \xrightarrow{hb} a \xrightarrow{mc} r1 \xrightarrow{dr} r2}$}%
\only<2>{$w \xrightarrow{hb} f \xrightarrow{hb} \textcolor{black!20}{a \xrightarrow{mc} r1 \xrightarrow{dr} r2}$}%
\only<3>{$w \xrightarrow{hb} f \xrightarrow{hb} a \xrightarrow{mc} \textcolor{black!20}{r1 \xrightarrow{dr} r2}$}%
\only<4->{$w \xrightarrow{hb} f \xrightarrow{hb} a \xrightarrow{mc} r1 \xrightarrow{dr} r2$}%
};
\end{tikzpicture}

% Сорцы и tikz-якоря
\begin{minipage}{0.49\textwidth}
	\begin{lstlisting}[escapechar=~]
T l = new T() {{
  fx = new int[1]; ~\tikz[na]\node[main node] (q) {q};~
  fx[0] = 42; ~\tikz[na]\node[target node] (w) {w};~
}}; ~\tikz[na]\node[main node] (f) {f};~
GLOBAL = l; ~\tikz[na]\node[main node] (a) {a};~
	\end{lstlisting}
\end{minipage}
\begin{minipage}{0.49\textwidth}
	\begin{lstlisting}[escapechar=~]
T o = GLOBAL;
if (o != null) {
  int[] lfx = o.fx; ~\tikz[na]\node[main node] (r1) {r1};~
  int result = lfx[0]; ~\tikz[na]\node[target node] (r2) {r2};~
}
	\end{lstlisting}
\end{minipage}

% Подписи, потихоньку появляющиеся под сорцами
\begin{center}
\only<2>{Действия в одном потоке образуют happens-before:\\ $w \hb f$, $f \hb a$}%
\only<3>{$r1$ видит запись (т.к. простое final поле):\\ $a \mc r1$}%
\only<4>{Поток 2 не создавал объект $q$, $r2$ читает 0ой элемент $q$, $r1$ это единственное чтение адреса объекта $q$, поэтому dereference chain:\\ $r1 \dr r2$}%
\only<5>{Нашли всё необходимое для $HB^*$: \\$w \hb f \hb a \mc r1 \dr r2 \Rightarrow w \hbs r2$}%
\only<6>{($w \hbs r2) \Rightarrow result \in \{42\}$ \\
Результат не зависит от того, в каком порядке выполнялись действия $(q)$ и $(w)$!}%
\end{center}

% Have no idea what it is, but it avoids repositioning of code blocks across slides
\vskip0pt plus 1filll

% Стрелочки между tikz-якорями
\begin{tikzpicture}[overlay]
  \path[hb edge,->]<2-> (w) edge [left] node[hb label] {hb} (f);
  \path[hb edge,->]<2-> (f) edge [bend left, looseness=0.1] node[hb label] {hb} (a);
  \path[mc edge,->]<3-> (a) edge [bend left, looseness=0.7] node[mc label] {mc} (r1);
  \path[dr edge,->]<4-> (r1) edge [bend right, out = 270, in = 270, looseness = 2] node[dr label] {dr} (r2);
  \path[hbs edge,->]<5-> (w) edge [bend right, out = 90, in = 90, looseness = 0.7] node[hbs label] {hb*} (r2);
\end{tikzpicture}

\end{frame}


\end{document}
