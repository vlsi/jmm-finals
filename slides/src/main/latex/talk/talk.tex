\documentclass[russian,aspectratio=169,14pt]{beamer}

\usepackage{listings}
\usepackage{tikz}
\usepackage{amsmath}
\usetikzlibrary{arrows,shapes}

\definecolor{mygreen}{rgb}{0,0.4,0}
\definecolor{myid}{rgb}{0.1,0.1,0.1}
\lstdefinestyle{Java}{language=java,
        numbers=none,stepnumber=1,numberstyle=\small\ttfamily,
        numbersep=5pt,extendedchars=true,
        commentstyle=\color{mygreen}\ttfamily,
        stringstyle=\color{magenta},
        keywordstyle=\color{violet}\bfseries,
        ndkeywordstyle=\color{yellow}\bfseries,
        identifierstyle=\color{myid},
        basicstyle=\small\ttfamily,
        escapechar=~,
        frame=none,
        lineskip=5pt,
        tabsize=4
}

\lstset{breakatwhitespace=false,
        language=Java,
        style=Java,
}

\definecolor{gold}{rgb}{0.85,.66,0}

\tikzstyle{na} = [baseline=-0.5ex]
\tikzstyle{main node} = [rectangle,fill=black!20,draw,font=\ttfamily\small\bfseries]
\tikzstyle{target node} = [rectangle,fill=gold!20,draw,font=\ttfamily\small\bfseries]
\tikzstyle{hb label} = [fill=green!30,font=\ttfamily\small\bfseries]
\tikzstyle{mc label} = [fill=red!30,font=\ttfamily\small\bfseries]
\tikzstyle{dr label} = [fill=blue!30,font=\ttfamily\small\bfseries]
\tikzstyle{hbs label} = [fill=gold!30,font=\ttfamily\small\bfseries]
\tikzstyle{hb edge} = [draw=green!30,thick,line width=2,font=\ttfamily\small\bfseries]
\tikzstyle{mc edge} = [draw=red!30,thick,line width=2,font=\ttfamily\small\bfseries]
\tikzstyle{dr edge} = [draw=blue!30,thick,line width=2,font=\ttfamily\small\bfseries]
\tikzstyle{hbs edge} = [draw=gold!30,thick,line width=2,font=\ttfamily\small\bfseries]

\newcommand{\hb}{\xrightarrow{\fcolorbox{green!30}{green!30}{\scriptsize hb}}}
\newcommand{\mc}{\xrightarrow{\fcolorbox{red!30}{red!30}{\scriptsize mc}}}
\newcommand{\dr}{\xrightarrow{\fcolorbox{blue!30}{blue!30}{\scriptsize dr}}}
\newcommand{\hbs}{\xrightarrow{\fcolorbox{gold!30}{gold!30}{\scriptsize $hb^*$}}}


\tikzstyle{every picture}+=[remember picture]

\begin{document}

% Это весь слайд целиком с анимацией
%
\begin{frame}[fragile]
\frametitle{$HB^*$ example 3}

\begin{tikzpicture}[overlay]
  \node [anchor=north east, inner sep=10pt] at (current page.north east)
     {
\only<1>{\textcolor{black!20}{$w \xrightarrow{hb} f \xrightarrow{hb} a \xrightarrow{mc} r1 \xrightarrow{dr} r2$}}
\only<2>{$w \xrightarrow{hb} f \xrightarrow{hb}$ \textcolor{black!20}{$a \xrightarrow{mc} r1 \xrightarrow{dr} r2$}}
\only<3>{$w \xrightarrow{hb} f \xrightarrow{hb} a \xrightarrow{mc}$ \textcolor{black!20}{$r1 \xrightarrow{dr} r2$}}
\only<4->{$w \xrightarrow{hb} f \xrightarrow{hb} a \xrightarrow{mc} r1 \xrightarrow{dr} r2$}
     };
\end{tikzpicture}

% Сорцы и tikz-якоря
\begin{minipage}{0.49\textwidth}
	\begin{lstlisting}[escapechar=~]
T l = new T() {{
  fx = new int[1]; ~\tikz[na]\node[main node] (q) {q};~
  fx[0] = 42; ~\tikz[na]\node[target node] (w) {w};~
}}; ~\tikz[na]\node[main node] (f) {f};~
GLOBAL = l; ~\tikz[na]\node[main node] (a) {a};~
	\end{lstlisting}
\end{minipage}
\begin{minipage}{0.49\textwidth}
	\begin{lstlisting}[escapechar=~]
T o = GLOBAL;
if (o != null) {
  int[] lfx = o.fx; ~\tikz[na]\node[main node] (r1) {r1};~
  int result = lfx[0]; ~\tikz[na]\node[target node] (r2) {r2};~
}
	\end{lstlisting}
\end{minipage}


% Подписи, потихоньку появляющиеся под сорцами
\begin{center}
\only<2>{Program order induces happens-before:\\ $w \hb f$, $f \hb a$}
\only<3>{$r1$ sees the write:\\ $a \mc r1$}
\only<4>{For thread 2, it did not init $q$, $r2$ reads 0-th element of $q$, $r1$ is the only read of $q$ address, therefore by "dereference chain" definition:\\ $r1 \dr r2$}
\only<5>{Taking everything, it follows from $HB^*$ definition: \\$(w \hb f) \land (f \hb a) \land (a \mc r1) \land (r1 \dr r2) \Rightarrow (w \hbs r2)$}
\only<6>{($w \hbs r2) \Rightarrow result \in \{42\}$ \\
Does not matter if $(q \xrightarrow{po} w)$ or $(w \xrightarrow{po} q)$!}
\end{center}

% Стрелочки между tikz-якорями
\begin{tikzpicture}[overlay]
  \path[hb edge,->]<2-> (w) edge [left] node[hb label] {hb} (f);
  \path[hb edge,->]<2-> (f) edge [left] node[hb label] {hb} (a);
  \path[mc edge,->]<3-> (a) edge [bend left, looseness=0.9] node[mc label] {mc} (r1);
  \path[dr edge,->]<4-> (r1) edge [bend right, out = 270, in = 270, looseness = 2] node[dr label] {dr} (r2);
  \path[hbs edge,->]<5-> (w) edge [bend right, out = 90, in = 90, looseness = 0.7] node[hbs label] {hb*} (r2);
\end{tikzpicture}

\end{frame}


\end{document}
