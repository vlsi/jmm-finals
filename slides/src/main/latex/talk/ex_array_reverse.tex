% ---------------------------------------
% Пример 2.1 "порядок присваивания final"
%
\ifrender
\subsection{Массив наоборот (2.1)}
\begin{frame}[fragile]{Массив наоборот (2.1)}
\vskiptitle
\hbsprogress{2-}{2-}{2-}{2-}%
\begin{minipage}{0.5\textwidth}
	\begin{lstlisting}
T l = new T() {{
  int[] u = new int[1];
  fx = u; // (!)
  u[0] = 42; ~\nd<1->[1,2,3]{w}~
}}; ~\nd<2->{f}~
GLOBAL = l; ~\nd<2->{a}~
	\end{lstlisting}
\end{minipage}%
\begin{minipage}{0.5\textwidth}
	\begin{lstlisting}
T o = GLOBAL;
if (o != null) {
  int[] lfx = o.fx; ~\nd<2->{r1}~
  int result = lfx[0]; ~\nd<2->[2,3]{r2}~
}
	\end{lstlisting}
\end{minipage}

\vskipfill

% Стрелочки между tikz-якорями
\begin{tikzpicture}[overlay]
  \path[hb edge,->]<2-> (w) edge [] node {hb} (f);
  \path[hb edge,->]<2-> (f) edge [bend left=15] node {hb} (a);
  \path[mc edge,->]<2-> (a) edge [bend left, looseness=0.7] node[mc label] {mc} (r1);
  \path[dr edge,->]<2-> (r1) edge [bend right, out = 270, in = 270, looseness = 2.2] node {dr} (r2);
  \path[hbs edge,->]<2-> (w) edge [bend right, out = 90, in = 90, looseness = 0.7] node {hb$^*$} (r2);
\end{tikzpicture}

% Подписи, потихоньку появляющиеся под сорцами
\begin{center}
\only<1>{Возможно ли в \result~получить $0$?}%
\only<2>{Строим {\hbsbox} так же как и в предыдущем случае}%
\only<3>{($w \hbs r2) \Rightarrow$ \result~$\in \{42\}$\\Результат не зависит от порядка записи final полей!}%
\end{center}

\vskipfill

\end{frame}
\fi
