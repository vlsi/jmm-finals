% ---------------------------------------
% Dereference chain. Пример возникновения
%
\ifrender
\subsection{Dereference chain}
\begin{frame}[fragile]{Dereference chain: что это?}
\hbsprogresss{0}{0}{0}{0}{0}{0}{0}{1-}{0}%
\vskiptitle
\begin{center}
\begin{minipage}[3]{.5\linewidth}
	\begin{lstlisting}
T local = GLOBAL; ~\nd<1->{r1}~

int localX = local.x; ~\nd<1->[1-]{r2}~
	\end{lstlisting}
\end{minipage}
\end{center}

\vskipfill

% Стрелочки между tikz-якорями
\begin{tikzpicture}[overlay]
  \path[dr edge,->]<4-> (r1) edge [bend right, out = 90, in = 90, looseness = 2] node {dr} (r2);
\end{tikzpicture}%

\begin{itemize}[<+->]
\item \nd[1-]{r2} читает поле объекта
\item Поток не создавал объект
\item Значит, где-то мы должны были читать адрес этого объекта
\item Это называется $r1 \dr r2$
\end{itemize}

\vskipfill

\end{frame}
\fi
